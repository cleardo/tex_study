\documentclass{article}
\usepackage{fontspec}
\usepackage{xeCJK}
\usepackage{tikz}
\usepackage{pgfplots}
\usepackage{pgfplotstable}
\usepackage{pgf-pie}
\usepackage{caption}
\usepackage{multirow}
\usepackage{float}

\pgfplotsset{compat=newest}
\definecolor{myblue}{RGB}{52, 152, 219}

\definecolor{blue01}{RGB}{41, 128, 185}

\definecolor{red01}{RGB}{192, 57, 43}
\definecolor{red02}{RGB}{255, 0, 151}

\definecolor{green01}{RGB}{22,160,133}

\usepackage[colorlinks = true,
            linkcolor = blue,
            urlcolor  = blue,
            citecolor = blue01,
            anchorcolor = blue01]{hyperref}

\pagestyle{empty}

\setCJKmainfont{SimSun}

\begin{document}

\newcommand{\tabincell}[2]{
    \begin{tabular}
        {@{}#1@{}}#2
    \end{tabular}
}

\newcommand\qaversion{V0.2.5}
\newcommand\qatesturl{http://debug.sdp.nd}
%\newcommand\qarequrl{http://demo.sdp.nd/SDP-V0.2\%20Design\%20By\%20Liyj/\#p=网龙开发者门户原型}
\newcommand\qarequrl{http://demo.sdp.nd/SDP-V0.2\%20Design\%20By\%20Liyj/}
\newcommand\qapmsurl{}

\title{共享平台{\qaversion}内网测试报告}
\author{QA部WEB测试三组共享平台项目组}
\maketitle


\colorbox{myblue}{\makebox(0.01, 9){\ }} 基本信息\\
\begin{table}[!hbp]
\renewcommand{\arraystretch}{2}
\begin{tabular}{|l|l|l|l|}
\hline
\textcolor{red02}{\textbf{平台}} & WEB端 & \textcolor{red02}{\textbf{版本号}} & {\qaversion} \\
\hline
\textcolor{red02}{\textbf{项目名称}} & 共享平台门户 & \textcolor{red02}{\textbf{测试地址}} & \url{\qatesturl} \\
\hline
\textcolor{red02}{\textbf{测试类型}} & 功能测试 & \textcolor{red02}{\textbf{测试范围}} & \href{\qarequrl}{参见需求文档} \\
\hline
\textcolor{red02}{\textbf{测试人员}} & 林斯潇、林志宏、杨晓燕 & 
\textcolor{red02}{\textbf{BUG详情}} & \href{http://}{点击查看} \\
\hline
\textcolor{red02}{\textbf{程序沟通人员}} & \tabincell{l}{
蔡扬兴、李书德、林龙钢、吴杰明\\
欧勇、陈文杰、温志杰、李辉\\
李鲁闽、林凌清、李勤、黄建新\\
翁雨辰、陈浩亮、叶金龙、胡淮波} & \textcolor{red02}{\textbf{用例执行率}} & 100\% \\
\hline
\textcolor{red02}{\textbf{需求沟通人员}} & 李永均 & \textcolor{red02}{\textbf{是否允许发布}} & 是 \\
\hline
\textcolor{red02}{\textbf{原型地址}} & \multicolumn{3}{|c|}{{\qarequrl}} \\
\hline
\multirow{2}*{\parbox{4em}{\textcolor{red02}{\textbf{测试环境}}}} & 操作系统 & \multicolumn{2}{|l|}{Windows7、Windows8} \\
\cline{2-4}
& 浏览器 & \multicolumn{2}{|l|}{Chrome、Firefox} \\
\hline
\end{tabular}
\end{table}

\par{}

\colorbox{myblue}{\makebox(0.01, 9){\ }} 测试结论\\
\begin{itemize}
\item 本次测试中主要的功能模块通过。
\item 截止报告时间,共提交BUG: 14个,建议: 6个,其中激活状态的BUG建议: 7个,延期处理: 0个,未解决的严重BUG数: 4个。
\item 截至报告时间,禅道上当前版本存在激活状态BUG: 7个。不影响功能,建议发布。
\end{itemize}


\colorbox{myblue}{\makebox(0.01, 9){\ }} BUG现状\\

\begin{table}[H]
\renewcommand{\arraystretch}{2}
\begin{tabular}{|c|c|c|c|c|}
\hline
BUG总数 & 致命BUG数 & 严重BUG数 & 一般BUG数 & 轻微BUG数 \\

\hline
14个 &0个 & 7个 & 7个 & 0\\
\hline
建议总数 & \multicolumn{4}{|l|}{6个} \\
\hline

\end{tabular}
\end{table}

\colorbox{myblue}{\makebox(0.01, 9){\ }} 严重程度统计\\

\begin{figure}[!hbp]

\pgfplotstableread{
    % 建议数 1轮 2轮 trunk主干
0 0
1 6
2 7
3 7

% Bug数
}\dataset

\begin{tikzpicture}

\begin{axis}[
    ybar,
    width=12cm,
    height=8cm,
    ymin=0,
    ymax=10,
    ylabel={Bug数},
    % 刻度
    xtick=data,
    xticklabels={
        \strut 建议,
        \strut 轻微,
        \strut 一般,
        \strut 严重,
        \strut 致命
    }]

\addplot[draw=black, fill=myblue, nodes near coords] table[x index=0, y index=1] \dataset;
\end{axis}
\end{tikzpicture}
\end{figure}


\colorbox{myblue}{\makebox(0.01, 9){\ }} BUG状态统计\\

\begin{figure}[H]
\begin{tikzpicture}
\begin{axis}[
symbolic x coords={激活, 已解决, 已关闭},
    xtick=data]
    \addplot[ybar, fill=red02, nodes near coords] coordinates {
(激活, 7)
(已解决, 10)
(已关闭, 3)

    };

\end{axis}
\end{tikzpicture}
\end{figure}

\colorbox{myblue}{\makebox(0.01, 9){\ }} BUG类型统计\\


\begin{figure}[H]
\begin{tikzpicture}
    \pie{
        10/配置相关, 20/设计缺陷, 5/安装部署, 50/代码错误, 15/界面优化  
    }
\end{tikzpicture}
\end{figure}

\colorbox{myblue}{\makebox(0.01, 9){\ }} 解决方案统计\\
\begin{figure}[H]\begin{tikzpicture}
    \pie{
        0/延期处理, 35/未解决, 5/设计如此, 55/已解决, 5/不予解决    
    }
\end{tikzpicture}
\end{figure}

\colorbox{myblue}{\makebox(0.01, 9){\ }} Bug数据\\

\begin{table}[H]
\renewcommand{\arraystretch}{1.5}
\begin{tabular}{|c|p{11cm}|c|}
\hline
ID & 标题 & 链接 \\
\hline
244 & 【web应用--生产环境发布】更新web应用代码后,生产环境-正式版本发布内容没有更新 & \href{http://pms.sdp.nd/index.php?m=bug\&f=view\&bugID=244}{查看详情}\\
\hline
243 & 【接口文档】 多处文字错误,建议修改 & \href{http://pms.sdp.nd/index.php?m=bug\&f=view\&bugID=243}{查看详情}\\
\hline
240 & 【接口文档】将接口名称写成操作名称 & \href{http://pms.sdp.nd/index.php?m=bug\&f=view\&bugID=240}{查看详情}\\
\hline
237 & 【API文档】文档中关于对象部分说明,一些注释信息跑到注释符号前,且排版不美观 & \href{http://pms.sdp.nd/index.php?m=bug\&f=view\&bugID=237}{查看详情}\\
\hline
236 & 【一键发布】发布过程页面布局不美观 & \href{http://pms.sdp.nd/index.php?m=bug\&f=view\&bugID=236}{查看详情}\\
\hline
235 & 【Eclipse开发环境】升级文档没有提供 & \href{http://pms.sdp.nd/index.php?m=bug\&f=view\&bugID=235}{查看详情}\\
\hline
234 & 【一键发布】灰度发布失败,但是还是有下载地址和二维码 & \href{http://pms.sdp.nd/index.php?m=bug\&f=view\&bugID=234}{查看详情}\\
\hline
233 & 【附加更多服务--创建服务】在从“附加更多服务”处进入后,创建服务,仅需要创建当前环境的服务 & \href{http://pms.sdp.nd/index.php?m=bug\&f=view\&bugID=233}{查看详情}\\
\hline
232 & 【发布类型说明建议】希望把发布改为正式发布 & \href{http://pms.sdp.nd/index.php?m=bug\&f=view\&bugID=232}{查看详情}\\
\hline
231 & 【灰度发布】链接指向war包 & \href{http://pms.sdp.nd/index.php?m=bug\&f=view\&bugID=231}{查看详情}\\
\hline
230 & 【添加伙伴】添加伙伴后的推送信息,门户地址错误 & \href{http://pms.sdp.nd/index.php?m=bug\&f=view\&bugID=230}{查看详情}\\
\hline
229 & 【灰度发布】灰度发布后,无法访问网站 & \href{http://pms.sdp.nd/index.php?m=bug\&f=view\&bugID=229}{查看详情}\\
\hline
228 & 【一键发布】历史版本中,无法区分是正式版本还是灰度发布版本 & \href{http://pms.sdp.nd/index.php?m=bug\&f=view\&bugID=228}{查看详情}\\
\hline
227 & 【一键发布】发布失败,但是失败原因缺失 & \href{http://pms.sdp.nd/index.php?m=bug\&f=view\&bugID=227}{查看详情}\\
\hline
226 & 【门户】删除应用出错 & \href{http://pms.sdp.nd/index.php?m=bug\&f=view\&bugID=226}{查看详情}\\
\hline
225 & 【应用创建】应用创建后99U推送svn地址是中包含IP地址,建议修改 & \href{http://pms.sdp.nd/index.php?m=bug\&f=view\&bugID=225}{查看详情}\\
\hline
224 & 【分布式存储服务信息】能不能不换行 & \href{http://pms.sdp.nd/index.php?m=bug\&f=view\&bugID=224}{查看详情}\\
\hline
223 & 【每日构建】构建信息没有通过99U进行推送 & \href{http://pms.sdp.nd/index.php?m=bug\&f=view\&bugID=223}{查看详情}\\
\hline
221 & 【域名】 web应用的域名不应该包含下划线 & \href{http://pms.sdp.nd/index.php?m=bug\&f=view\&bugID=221}{查看详情}\\
\hline
197 & 登录时勾选记住密码,关闭浏览器后,下次打开还是未登录状态,要输密码登录。 & \href{http://pms.sdp.nd/index.php?m=bug\&f=view\&bugID=197}{查看详情}\\
\hline
\end{tabular}
\end{table}

\colorbox{myblue}{\makebox(0.01, 9){\ }} 风险评估\\

1. 浏览器兼容性优先级测试如下:IE8、IE11、谷歌、360,其它浏览器和系统分辨率没有进行兼容性测试,可能存在兼容性风险

2. 本次跟进只对跟进范围进行测试,故存在漏测风险。

3. 本轮测试在网络正常的情况下测试的,未对网络异常情况下进行测试,故发布存在风险。

4. 本轮测试在没有模拟所有系统、软件和硬件环境融合情况下进行测试,故发布存在风险。

5. 本轮测试只在内网环境中测试,未同步测试外网的功能,故发布存在风险。

\colorbox{myblue}{\makebox(0.01, 9){\ }} 存在问题\\

\end{document} 
